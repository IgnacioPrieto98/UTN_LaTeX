%---------------FORMATO VISUAL DE TEXTO---------------------------%

%---OBJETIVO---%

% Mostrar el uso de los diferentes tamaños de texto que provee LaTeX.

%-------------------------------------------------------------------%

% Al diseñar un texto es importante poseer diferentes posibilidades de 
% formas de resaltar algunas partes características del mismo, como: 
% tamaños de fuente, texto en negrita, texto inclinado, texto subrayado, 
% entre otras. Los códigos que permiten alterar el tamaño de la fuente son
% los siguientes:
%	-Fuente enana:	 			{\tiny   }
%	-Fuente tanaño índice: 		{\scriptsize   }
%	-Fuente tamaño nota de pie: {\footnotesize   }
%	-Fuente pequeña:			{\small   }
%	-Fuente grande:				{\large   }
%	-Fuente muy grande:			{\Large   }
%	-Fuente aún más grande:		{\LARGE   }
%	-Fuente enorme:				{\huge   }
%	-Fuente de mayor tamaño:	{\Huge   }

% Es notable que cada modificador posee un juego de llaves luego de la línea de 
% código que la representa. Esto es así debido a que allí se debe escribir el texto
% que cambiará su tamaño.

% El tamaño del texto que se altera es relativo al tamaño definido dentro del
% documento en el \documentclass{}. 

% Por otro lado, los códigos que permiten alterar la posición o el formato
% de la fuente son:
%	-Fuente negrita:	\textbf{}
%	-Fuente subrayada: 	\underline{}
%	-Fuente inclindado: \textit{}
%	-Fuente enfatizada:	\emph{}
%	-Fuente con mayusc: \scshape{}


\documentclass[12pt]{article}

\begin{document}
	
	\tiny{Texto comparativo de tamaño}
	
	\scriptsize{Texto comparativo de tamaño}
	
	\footnotesize{Texto comparativo de tamaño}
	
	\small{Texto comparativo de tamaño}
	
	Texto comparativo de tamaño
	
	{\large Texto comparativo de tamaño}
	
	{\Large Texto comparativo de tamaño}
	
	{\LARGE Texto comparativo de tamaño}
	
	{\huge Texto comparativo de tamaño}
	
	{\Huge Texto comparativo de tamaño}
	
		
	\textbf{Texto comparativo de fuente}
	
	\underline{Texto comparativo de fuente}
	
	\textit{Texto comparativo de fuente}
	
	\emph{Texto comparativo de fuente}
	
	\scshape{Texto comparativo de fuente}

\end{document}