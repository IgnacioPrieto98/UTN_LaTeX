%----------------PARÁMETROS DEL DOCUMENTO---------------------------%

%---OBJETIVO---%

% Mostrar el uso de jerarquías de títulos y subtítulos en LaTeX.

%-------------------------------------------------------------------%

% Al trabajar con LaTeX, el uso de jerarquías resulta importante debido
% a que mantiene ordenado el contenido y con una lógica fácil de entender
% y seguir. Las jerarquías de LaTeX son las siguientes:
%	-Capítulos: 		\chapter{}
%	-Secciones: 		\section{}
%	-Subsecciones: 		\subsection{}
%	-Subsubseciones: 	\subsubsection{}	

% Es notable que cada jerarquía posee un juego de llaves luego de la línea de 
% código que la representa. Esto es así debido a que allí se debe escribir
% el nombre que llevará dicha jerarquía.

% Por defecto en latex estas jerarquías contienen una enumeración dependiendo
% la posición en la que se ubican que alternan automáticamente al moverse.
% Además, poseen un formato (el cual en otro ejemplo se verá como personalizar)
% que caracteríza a cada una.


\documentclass[a4]{book}

\begin{document}
	
	\chapter{Animales}
	%Texto de relleno sin sentido.
	Lorem ipsum dolor sit amet, consectetur adipiscing elit, sed do eiusmod tempor incididunt ut labore et dolore magna aliqua. Enim nulla aliquet porttitor lacus. Eu consequat ac felis donec et. Augue ut lectus arcu bibendum at. Donec ultrices tincidunt arcu non sodales neque sodales
	
	\section{Vertebrados}
	%Texto de relleno sin sentido.
	Lorem ipsum dolor sit amet, consectetur adipiscing elit, sed do eiusmod tempor incididunt ut labore et dolore magna aliqua. Enim nulla aliquet porttitor lacus. Eu consequat ac felis donec et. Augue ut lectus arcu bibendum at. Donec ultrices tincidunt arcu non sodales neque sodales
	
	\subsection{Mamíferos}
	%Texto de relleno sin sentido.
	Lorem ipsum dolor sit amet, consectetur adipiscing elit, sed do eiusmod tempor incididunt ut labore et dolore magna aliqua. Enim nulla aliquet porttitor lacus. Eu consequat ac felis donec et. Augue ut lectus arcu bibendum at. Donec ultrices tincidunt arcu non sodales neque sodales
	
	\subsection{Reptiles}
	%Texto de relleno sin sentido.
	Lorem ipsum dolor sit amet, consectetur adipiscing elit, sed do eiusmod tempor incididunt ut labore et dolore magna aliqua. Enim nulla aliquet porttitor lacus. Eu consequat ac felis donec et. Augue ut lectus arcu bibendum at. Donec ultrices tincidunt arcu non sodales neque sodales
	
	
	\section{Invertebrados}
	%Texto de relleno sin sentido.
	Lorem ipsum dolor sit amet, consectetur adipiscing elit, sed do eiusmod tempor incididunt ut labore et dolore magna aliqua. Enim nulla aliquet porttitor lacus. Eu consequat ac felis donec et. Augue ut lectus arcu bibendum at. Donec ultrices tincidunt arcu non sodales neque sodales
	
	\subsection{Artrópodos}
	%Texto de relleno sin sentido.
	Lorem ipsum dolor sit amet, consectetur adipiscing elit, sed do eiusmod tempor incididunt ut labore et dolore magna aliqua. Enim nulla aliquet porttitor lacus. Eu consequat ac felis donec et. Augue ut lectus arcu bibendum at. Donec ultrices tincidunt arcu non sodales neque sodales
	
		\subsubsection{Insectos}
		%Texto de relleno sin sentido.
		Lorem ipsum dolor sit amet, consectetur adipiscing elit, sed do eiusmod tempor incididunt ut labore et dolore magna aliqua. Enim nulla aliquet porttitor lacus. Eu consequat ac felis donec et. Augue ut lectus arcu bibendum at. Donec ultrices tincidunt arcu non sodales neque sodales
		
		\subsubsection{Arácnidos}
		%Texto de relleno sin sentido.
		Lorem ipsum dolor sit amet, consectetur adipiscing elit, sed do eiusmod tempor incididunt ut labore et dolore magna aliqua. Enim nulla aliquet porttitor lacus. Eu consequat ac felis donec et. Augue ut lectus arcu bibendum at. Donec ultrices tincidunt arcu non sodales neque sodales
		
	
\end{document}