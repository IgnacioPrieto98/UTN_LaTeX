%----------POSICIONAMIENTO Y DISTANCIAS--------------%

%---OBJETIVO---%

% Aprender como centrar, justificar, alinear los lados 
% el texto.
% 
% Aprender a separar elementos en distancias definidas
% verticales y horizontales.

%---------------------------------------------------%

% Usualmente en los textos se utiliza el cambio de pos.
% del texto para realizar énfasis en lo que se está des-
% plazando; ya sea un título, un subtítulo, una ecuación
% una tabla, etc. Para poder realizar esto en un documento
% es posible utilizar entornos o líneas de código. 
% 
% La diferencia entre trabajar con entornos y con líneas
% de código no es amplia para realizar el desplazamiento 
% en si. Aun así, se debe saber que si se trabaja con
% entornos, todo lo que se aplique dentro de estos solo
% afectará a lo que esté adentro del mismo. Permitiendo 
% modificar fácilmente el contenido del entorno. Acción
% que no es posible mediante la línea de código. 
% A continuación se muestra como realizar ambas.  

% Para realizar el desplazamiento del texto por entornos
% se recurre a los siguientes nombres de entorno:
% 	- Centrado: center
% 	- Desplazado a la izquierda: flushleft
% 	- Desplazado a la derecha: flushright
% Donde el texto justificado no se define en ningún entorno
% debido a que es el predeterminado de LaTeX.

% Se debe recordar que para utilizar un entorno se debe
% abrir el mismo mediante \begin{nombre-del-entorno} y 
% luego cerrarlo mediante \end{nombre-del-entorno}.

% Por otro lado, si se requiere utilizar líneas de código,
% se aplica:
%	-Centrado: \centering
%	-Desplazado a la izquierda: \raggedleft
%	-Desplazado a la derecha: \raggedright
% Donde nuevamente el texto justificado no se define mediante
% ningún comando ya que es el predeterminado de LaTeX.
 
% Aquí se debe tener en cuenta que el texto se modificará
% a partir de la línea en la que se presenta el código para 
% abajo, hasta que exista un cambio de posicionamiento. 

%Además, como no se presenta comando para volver a justificar
%el texto (al menos no sin paquetes) todo el documento quedará
%con el último posicionamiento seleccionado.
 	
% Una última diferencia entre ambos casos, es que al utilizar 
% entornos, se comienza siempre un nuevo párrafo. Por otro lado,
% al utilizar las líneas de comando, no.

\documentclass{article}

\begin{document}

	\begin{center}
		Lorem ipsum dolor sit amet, consectetur adipiscing elit, sed do eiusmod tempor incididunt ut labore et dolore magna aliqua. At augue eget arcu dictum. Posuere morbi leo urna molestie at.
	\end{center}	

	\begin{flushleft}
		Lorem ipsum dolor sit amet, consectetur adipiscing elit, sed do eiusmod tempor incididunt ut labore et dolore magna aliqua. At augue eget arcu dictum. Posuere morbi leo urna molestie at.
	\end{flushleft}
	
	\begin{flushright}
		Lorem ipsum dolor sit amet, consectetur adipiscing elit, sed do eiusmod tempor incididunt ut labore et dolore magna aliqua. At augue eget arcu dictum. Posuere morbi leo urna molestie at.
	\end{flushright}
	
	
	Lorem ipsum dolor sit amet, consectetur adipiscing elit, sed do eiusmod tempor incididunt ut labore et dolore magna aliqua. At augue eget arcu dictum. Posuere morbi leo urna molestie at.
	
	\raggedright
	Lorem ipsum dolor sit amet, consectetur adipiscing elit, sed do eiusmod tempor incididunt ut labore et dolore magna aliqua. At augue eget arcu dictum. Posuere morbi leo urna molestie at.
	
	\raggedleft
	Lorem ipsum dolor sit amet, consectetur adipiscing elit, sed do eiusmod tempor incididunt ut labore et dolore magna aliqua. At augue eget arcu dictum. Posuere morbi leo urna molestie at.
	
	\centering
	Lorem ipsum dolor sit amet, consectetur adipiscing elit, sed do eiusmod tempor incididunt ut labore et dolore magna aliqua. At augue eget arcu dictum. Posuere morbi leo urna molestie at.
	

	
\end{document} 
 