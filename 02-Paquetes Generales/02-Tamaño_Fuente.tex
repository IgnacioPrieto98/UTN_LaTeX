%---------------------TAMAÑO DE FUENTE---------------------------%

%---OBJETIVO---%
					   
% Conocer el paquete que permite alterar el tamaño de fuente de texto.

%-------------------------------------------------------------------%


% Para alterar un la altura de un texto en un documento LaTeX puede no
% bastar con los valores de alturas por defecto. Debido a esto se recurre
% al paquete:
% 
%
%		 					anyfontsize
%
%
% el cual mediante el comando 
% 
% 
% 				   \fontsize{~tamaño~}{~separación~}
%
%
% permite definir un tamaño de texto a partir de dos distancias.
%
% Para seleccionar el texto el cual se quiere alterar, se define posterior a 
% \fontsize el comando
% 
% 				    \selectfont{~texto a modificar~}
% 
% donde en las llaves se indica el texto que se desea modificar.
	
\documentclass{article}

%Llamado del paquete.
\usepackage{anyfontsize}

\begin{document}

%Texto de relleno.
Lorem ipsum dolor sit amet, consectetur adipiscing elit. 
Sed id sagittis nulla. Pellentesque venenatis molestie viverra. 
Morbi tincidunt mauris odio, ut blandit quam eleifend et. 
Integer eu augue nec sem sagittis euismod. 
Nullam dapibus, mauris non pretium malesuada, risus justo consectetur libero, non aliquet urna turpis eu dolor. 


%Ejemplo de aplicación del paquete

%Uso del comando \fontsize y \selectfont
\fontsize{1cm}{1cm} \selectfont{

%Texto relleno alterado
Lorem ipsum dolor sit amet, consectetur adipiscing elit. 
Sed id sagittis nulla. Pellentesque venenatis molestie viverra. 
Morbi tincidunt mauris odio, ut blandit quam eleifend et. 
Integer eu augue nec sem sagittis euismod. 
Nullam dapibus, mauris non pretium malesuada, risus justo consectetur libero, non aliquet urna turpis eu dolor. 


}

\end{document}

