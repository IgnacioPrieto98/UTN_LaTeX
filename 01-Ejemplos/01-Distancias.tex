%---------------DEFINICIÓN DE DISTANCIAS---------------%

%---OBJETIVO---%

% Conocer las distancias existentes en LaTeX referidas al documento
% y trabajar con ellas.

%---------------------------------------------%

% Usualmente, al querer trabajar con un documento es deseado conocer
% y definir ciertas distancias entre elementos como:
% 	- Espaciado vertical: \baselineskip
% 	- Distancia entre columnas: \columnsep (caso de múltiples columnas).
% 	- Ancho de columnas: \columnwidth (caso de múltiples columnas).
% 	- Ancho de línea: \linewidth (del entorno actual).
% 	- Ancho de margen: \oddsidemargin (usualmente en documentos de dos lados).
% 	- Ancho de página: \paperwidth
% 	- Alto de página: \paperheight
% 	- Sangría de párrafo: \parindent
% 	- Distancia entre párrafos: \parskip
% 	- Altura del texto: \textheight (\paperheight - margenes verticales).
% 	- Ancho del texto: \textwidth (\paperwidth - margenes horizontales).
% 	- Altura del margen superior: \topmargin
% 
% Así como existen estas distancias, pueden crearse distancias independientes
% o dependientes de otras existentes. 
% 
% Las distancias, como en cotidianidad, poseen unidades para ser medidas y 
% definidas. Las unidades usuales en LaTeX son:
% 	- puntos: pt
% 	- milímetros: mm
% 	- centímetros: cm
% 	- pulgadas: in
% 	- distancia x: ex (ancho de una x en "minúscula" en la fuente actual).
% 	- distancia m: em (ancho de una M en "mayúscula" en la fuente actual).
% entre otros.
% 
% La forma de definir una distancia se realiza mediante el comando
%
% \newlength{\nombre}
%
% Para definir su valor solo basta con el comando
% 
% \setlength{\nombre}{distancia}
%
% o si se busca que la distancia sea dependiente de otra
%
% \setlength{\nombre}{0.5\textwidth}
%
% Así como se pueden definir distancias nuevas, también pueden 
% modificarse algunas de las existentes mediante el comando
% \setlength. Estas son:
%
% 	-\baselineskip
%	-\columnsep
%	-\columnwidth
%	-\oddsidemargin
%	-\parskip
%	-\parindent
%	-\topmargin
%	
% El ancho de la página y el largo de la misma no pueden modificarse alternado
% las distancias respectivas al papel. De todas maneras, luego se podrá ver que
% esto no es inalterable.
%
% Por último, si se quiere trabajar con una relación algebraica dentro de la 
% definición de la distancia, es posible mediante el uso del siguiente grupo de
% comandos
% 
% \setlength{\nombre}{\dimexpr Expresión a utilizar \relax}
%
% A continuación se ven algunos ejemplos aplicados sobre un texto de referencia.
 
% Definición de un documento de doble lado y dos columnas.
\documentclass[twocolumn,twoside]{article}

% Modificación del largo y el ancho del texto.
\setlength{\textwidth}{5in}
\setlength{\textheight}{5in}

% Este par de distancias son las únicas que deben definirse
% obligatoriamente antes de iniciar el entorno de documento.
% Ya que al cargar el mismo, define los valores previamente.

% Definición de una distancia
\newlength{\distanciaA}
\setlength{\distanciaA}{1in}
 

\begin{document}
	
% Definición de una distancia

\newlength{\distanciaB}
\setlength{\distanciaB}{1in}

	
% Redefinición de distancias existentes.
\setlength{\baselineskip}{\dimexpr(0.1\distanciaA +0.2\distanciaB)\relax}
\setlength{\columnsep}{0.5in}
\setlength{\columnwidth}{3in}
\setlength{\oddsidemargin}{0.1in}
\setlength{\parskip}{0.01\columnwidth}
\setlength{\parindent}{2in}
\setlength{\topmargin}{\dimexpr(2in+\distanciaA/\columnwidth)\relax}

Texto de relleno - Texto de relleno - Texto de relleno - Texto de relleno - Texto de relleno - Texto de relleno - Texto de relleno - Texto de relleno - Texto de relleno - Texto de relleno - Texto de relleno - Texto de relleno - Texto de relleno - Texto de relleno - Texto de relleno - Texto de relleno - Texto de relleno - Texto de relleno - Texto de relleno - Texto de relleno - Texto de relleno - Texto de relleno - Texto de relleno - Texto de relleno - Texto de relleno - Texto de relleno - Texto de relleno - Texto de relleno - Texto de relleno - Texto de relleno - Texto de relleno - Texto de relleno - Texto de relleno - Texto de relleno - Texto de relleno - Texto de relleno - Texto de relleno - Texto de relleno - Texto de relleno - Texto de relleno - Texto de relleno - Texto de relleno - Texto de relleno - Texto de relleno - Texto de relleno - Texto de relleno - Texto de relleno - Texto de relleno - 


Texto de relleno - Texto de relleno - Texto de relleno - Texto de relleno - Texto de relleno - Texto de relleno - Texto de relleno - Texto de relleno - Texto de relleno - Texto de relleno - Texto de relleno - Texto de relleno - Texto de relleno - Texto de relleno - Texto de relleno - Texto de relleno - Texto de relleno - Texto de relleno - Texto de relleno - Texto de relleno - Texto de relleno - Texto de relleno - Texto de relleno - Texto de relleno - Texto de relleno - Texto de relleno - Texto de relleno - Texto de relleno - Texto de relleno - Texto de relleno - Texto de relleno - Texto de relleno - Texto de relleno - Texto de relleno - Texto de relleno - Texto de relleno - Texto de relleno - Texto de relleno - Texto de relleno - Texto de relleno - Texto de relleno - Texto de relleno - Texto de relleno - Texto de relleno - Texto de relleno - Texto de relleno - Texto de relleno - Texto de relleno -


Texto de relleno - Texto de relleno - Texto de relleno - Texto de relleno - Texto de relleno - Texto de relleno - Texto de relleno - Texto de relleno - Texto de relleno - Texto de relleno - Texto de relleno - Texto de relleno - Texto de relleno - Texto de relleno - Texto de relleno - Texto de relleno - Texto de relleno - Texto de relleno - Texto de relleno - Texto de relleno - Texto de relleno - Texto de relleno - Texto de relleno - Texto de relleno - Texto de relleno - Texto de relleno - Texto de relleno - Texto de relleno - Texto de relleno - Texto de relleno - Texto de relleno - Texto de relleno - Texto de relleno - Texto de relleno - Texto de relleno - Texto de relleno - Texto de relleno - Texto de relleno - Texto de relleno - Texto de relleno - Texto de relleno - Texto de relleno - Texto de relleno - Texto de relleno - Texto de relleno - Texto de relleno - Texto de relleno - Texto de relleno -

\end{document}