% xelatex 
\documentclass[oneside]{book}
\usepackage{lipsum}
\usepackage{STY-UTN-LTX-CATEDRAS}

\def\supervisor{Supervisor}
\def\titulo{Título del documento}
\def\subtitulo{Subtítulo del documento}
\def\autor{
	Autor de documento 1\\
	Autor de documento 2
	}

\dominitoc

%---AYUDA DE COMANDOS---%

%BEGIN_FOLD
%-----------------------%
% \Bib{-Bibliografía-}
%
% Uso:
% 
% El comando define el contenido que se presenta
% por debajo de la sección, cual indica la bibliografía
% de consulta. 
%	
% Ubicación:
% 
% El comando se presenta por encima del llamado 
% al comando "\section{}".
% 
% Dato:
% 
% Es importante definir este cada vez que se llame a
% una sección diferente.
%
%-----------------------%
% \Res{-Bibliografía-}
%
% Uso:
%
% El comando define el resumen presente en el capítulo.
%	
% Ubicación:
% 
% El comando se presenta por encima del llamado 
% al comando "\chapter{}".
% 
% Es importante definir este cada vez que se llame a
% una sección diferente.
%-----------------------%
%END_FOLD

\begin{document}
	
	\begin{titlepage}
			\begin{tikzpicture}[remember picture,overlay,node distance=2mm]
			\fill[C5] ($ (current page.north east)-(.8\marginparwidth,0)$) rectangle ($ (current page.south east)$);
		\end{tikzpicture}
		\thispagestyle{empty}
		\begin{flushleft}
			{\fontsize{40}{20}\fontspec{Montserrat} \titulo}\\
			\vspace{.5cm}
			{\fontsize{15}{20} \fontspec{Montserrat} \subtitulo}
		\end{flushleft}
		\vfill
		\includegraphics[width=0.9\textwidth]{titulo}
		\vspace{-1cm}
		\newpage
		
		\begin{tikzpicture}[remember picture,overlay,node distance=2mm]
			\fill[C5] ($ (current page.north east)-(.8\marginparwidth,0)$) rectangle ($ (current page.south east)$);
		\end{tikzpicture}
		
		{\color{white} UTN}
		
		\vfill
		
		\begin{flushright}
			\begin{minipage}{0.4\textwidth}
			{\fontspec{Montserrat}
				\hrule
				\vspace{2mm}
				\begin{flushright}
					\textbf{Fecha de finalización}\\
				\today\\
				\end{flushright}
				\hrule
				\vspace{2mm}
				\begin{flushright}
					\textbf{Desarrolladores}\\
				\autor\\
				\end{flushright}
				\hrule
				\vspace{2mm}
				\begin{flushright}
					\textbf{Docente supervisor}\\
				\supervisor\\
				\end{flushright}
				\hrule
				\vspace{2mm}
				\begin{flushright}
					\textbf{Localidad}\\
				Reconquista Santa Fe\\
				\end{flushright}
				\hrule
			}
			\end{minipage}
		\end{flushright}
		
	\end{titlepage}
	
	\tableofcontents

	\pagestyle{HojaContenido}
	
	\chapter{Título de capítulo}
	
\end{document}